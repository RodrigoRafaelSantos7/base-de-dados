\documentclass{article}
\usepackage{amsthm}


\makeatletter
\def\lecture{\@ifnextchar[{\@lectureWith}{\@lectureWithout}}
\def\@lectureWith[#1]{\medbreak\refstepcounter{section}%
  \renewcommand{\leftmark}{Lecture \thesection}
  \noindent{\addcontentsline{toc}{section}{Lecture \thesection: #1\@addpunct{.}}%
  \sectionfont Lecture \thesection. #1\@addpunct{.}}\medbreak}
\def\@lectureWithout{\medbreak\refstepcounter{section}%
  \renewcommand{\leftmark}{Lecture \thesection}
  \noindent{\addcontentsline{toc}{section}{Lecture \thesection.}%
  \sectionfont Lecture \thesection.}\medbreak}
\makeatother

\newcommand*{\xdash}[1][3em]{\rule[0.5ex]{#1}{0.55pt}}

\title{Base de Dados}
\author{Rodrigo Santos}
\begin{document}
\maketitle

\lecture[Diagramas ER]
\begin{itemize}
  \item \textbf{Retângulos}: representam conjuntos de entidades.
  \item \textbf{Losangos}: representam conjuntos de relações.
  \item \textbf{Elipses}: representam atributos.\textbf{Vamos fazer as coisas por forma a ter sempre atributos simples, uni-valor e não derivados}.
        \begin{itemize}
          \item Atributo \textbf{simples}.
          \item Atributo \textbf{composto}: Composto por vários atributos simples.
          \item Atributo \textbf{uni-valor e multi-valor}: exemplo de um atributo multi-valor: número\textbf{s} de telefone.
          \item Atributo \textbf{derivado}:  podem ser calculado a partir de outros atributos, como por exemplo a idade, a partir da data de nascimento.
        \end{itemize}
  \item \textbf{Linhas}: ligam atributos aos conjuntos de entidades e conjuntos de entidades a conjuntos de associações.
\end{itemize}

\textbf{Restrições de mapeamento (entre o conjunto de relações e o conjunto de entidades)}
\begin{itemize}
  \item \textbf{Seta} ($\rightarrow$): significa 1.
  \item \textbf{Linha} (\xdash[1em]): significa muitos.
\end{itemize}

\textbf{Participação}
\begin{itemize}
  \item \textbf{Participação total (=)}: indicado por uma linha dupla, significa que toda a entidade do conjunto de entidades participa em pelo menos uma relação do conjunto de relações.
  \item \textbf{Participação parcial}: algumas entidades podem não participar em qualquer relação do conjunto de relações.
\end{itemize}

\textbf{Conjunto de entidades fraco}
\begin{itemize}
  \item Um conjunto de entidades fraco é representado por um retângulo duplo.
  \item O atributo discriminante do conjunto de entidades fraco é sublinhado a tracejado.
  \item As relações são representadas por um retângulo duplo. (A relação entre o conjunto fraco e o dominante.)
\end{itemize}

\textbf{Especialização/Generalização (ISA)}
\begin{itemize}
  \item \textbf{Herança de atributos}: um conjunto de entidades de menor nível herda todos os atributos e participa em todas as relações do conjunto de entidades de maior nível. Por outras palavras herdam todos os atributos dos que estão acima deles, e por isso não necessitamos de voltar a repetir os atributos.
  \item \textbf{Disjunta}: só pode pertencer a um elemento do nível inferior. Representa-se com um \textbf{disjoint} ao lado do triângulo do ISA.
  \item \textbf{Sobreposição}: pode pertencer a vários elementos do nível inferior.
  \item \textbf{Total}: tem de pertencer a pelo menos um elemento do nível inferior. Representa-se com um linha dupla (Como a Participação total.)
  \item \textbf{Parcial}: pode não pertencer a nenhum.
\end{itemize}

\textbf{Chaves}
\begin{itemize}
  \item \textbf{Super-chave de um conjunto de entidades}: é um conjunto de um ou mais atributos cujos valores determinam univocamente cada uma das entidades dentro do conjunto. Uma super-chave pode ter informação desnecessária.
  \item \textbf{Chave candidata}: Uma chave candidata de um conjunto de entidades é uma super-chave minimal.
  \item \textbf{Chave primaria}: Chave primária é uma chave candidata designada para identificar as entidades dum conjunto.
  \item A combinação das chaves primárias dos conjuntos de entidades participantes formam uma super-chave do conjunto de relações. \textbf{isto significa que um par de entidades pode aparecer no máximo uma vez num conjunto de relações}.
  \item Temos que considerar a cardinalidade de mapeamento (Restrições de mapeamento mais acima) dos conjuntos de entidades quando decidimos quais as chaves candidatas dos conjuntos de relações.
\end{itemize}

\lecture[Modelo Relacional]
Os diagramas ER ajudam na modelação dos dados mas não ajudam como modelo para armazenar dados e no seu posterior tratamento.


\end{document}